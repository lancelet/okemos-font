\usemodule[tabl]

\setupexternalfigures[directory=.,conversion=pdf]

%----- GLYPH DIMENSION FIGURE -----
\define[1]\GlyphFig{%
    \placefigure[here][fig:#1]{\tt #1}{%
        \externalfigure[glyphdims/#1.svg][width=0.75\textwidth,frame=on]
    }
}

\starttext

\chapter[chp:glyph-designs]{Glyph Designs at Original Weight}

\section[sec:latin-capitals]{Latin Capitals}

Table \in{Table}{}[tab:latin-capitals] describes the origin of the glyphs for the latin capital letters.

\placetable[here][tab:latin-capitals]
    {Definitions of Latin capital letter glyphs.}{
    \starttabulate[|l|l|]
        \NC {\bf Character} \NC {\bf Definition} \NC\NR
        \HL
        \NC A \NC \in{Figure}{}[fig:latin-capital-letter-a] \NC\NR
        \NC B \NC \in{Figure}{}[fig:latin-capital-letter-b] \NC\NR
        \NC C \NC \in{Figure}{}[fig:latin-capital-letter-c] \NC\NR
        \NC D \NC Follows the design of B \NC\NR
        \NC E \NC \in{Figure}{}[fig:latin-capital-letter-e] \NC\NR
        \NC F \NC Follows the design of E \NC\NR
    \stoptabulate
}


\GlyphFig{latin-capital-letter-a}
\GlyphFig{latin-capital-letter-b}
\GlyphFig{latin-capital-letter-c}
\GlyphFig{latin-capital-letter-e}

\stoptext